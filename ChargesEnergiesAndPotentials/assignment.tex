%%%%%%%%%%%%%%%%%%%%%%%%%%%%%%%%%%%%%%%%%
% Short Sectioned Assignment
% LaTeX Template
% Version 1.0 (5/5/12)
%
% This template has been downloaded from:
% http://www.LaTeXTemplates.com
%
% Original author:
% Frits Wenneker (http://www.howtotex.com)
%
% License:
% CC BY-NC-SA 3.0 (http://creativecommons.org/licenses/by-nc-sa/3.0/)
%
%%%%%%%%%%%%%%%%%%%%%%%%%%%%%%%%%%%%%%%%%

%----------------------------------------------------------------------------------------
%	PACKAGES AND OTHER DOCUMENT CONFIGURATIONS
%----------------------------------------------------------------------------------------

\documentclass[paper=a4, fontsize=11pt]{scrartcl} % A4 paper and 11pt font size

\usepackage[T1]{fontenc} % Use 8-bit encoding that has 256 glyphs
\usepackage{fourier} % Use the Adobe Utopia font for the document - comment this line to return to the LaTeX default
\usepackage[english]{babel} % English language/hyphenation
\usepackage{amsmath,amsfonts,amsthm} % Math packages

\usepackage{lipsum} % Used for inserting dummy 'Lorem ipsum' text into the template

\usepackage{sectsty} % Allows customizing section commands
\allsectionsfont{\centering \normalfont\scshape} % Make all sections centered, the default font and small caps

\usepackage{fancyhdr} % Custom headers and footers
\pagestyle{fancyplain} % Makes all pages in the document conform to the custom headers and footers
\fancyhead{} % No page header - if you want one, create it in the same way as the footers below
\fancyfoot[L]{} % Empty left footer
\fancyfoot[C]{} % Empty center footer
\fancyfoot[R]{\thepage} % Page numbering for right footer
\renewcommand{\headrulewidth}{0pt} % Remove header underlines
\renewcommand{\footrulewidth}{0pt} % Remove footer underlines
\setlength{\headheight}{13.6pt} % Customize the height of the header

\numberwithin{equation}{section} % Number equations within sections (i.e. 1.1, 1.2, 2.1, 2.2 instead of 1, 2, 3, 4)
\numberwithin{figure}{section} % Number figures within sections (i.e. 1.1, 1.2, 2.1, 2.2 instead of 1, 2, 3, 4)
\numberwithin{table}{section} % Number tables within sections (i.e. 1.1, 1.2, 2.1, 2.2 instead of 1, 2, 3, 4)

\setlength\parindent{0pt} % Removes all indentation from paragraphs - comment this line for an assignment with lots of text

%----------------------------------------------------------------------------------------
%	TITLE SECTION
%----------------------------------------------------------------------------------------

\newcommand{\horrule}[1]{\rule{\linewidth}{#1}} % Create horizontal rule command with 1 argument of height

\title{	
\normalfont \normalsize 
\textsc{Sapporo, Department of Undefined Studies} \\ [25pt] % Your university, school and/or department name(s)
\horrule{0.5pt} \\[0.4cm] % Thin tp horizontal rule
\huge Charges, Energies and Potentials\\ % The assignment title
\horrule{2pt} \\[0.5cm] % Thick bottom horizontal rule
}

\author{Tobias Kuhn} % Your name

\date{\normalsize\today} % Today's date or a custom date

\begin{document}

\maketitle % Print the title

%----------------------------------------------------------------------------------------
%	PROBLEM 1
%----------------------------------------------------------------------------------------

\section{ Problem One }

\begin{align} 
Q_A &= -1[C] & W_A &= 60[J] \\
Q_B &= -2[C] & W_B &= 10[J] 
\end{align}

Two points A and B exist in a vacuum. They have energies and charges as describes above.

c) How much work is necessary to get 2.5 trillion electrons from infinitely far away to Point A?

We start our investigation by calculating the electrostatic charge of $2.5 \cdot 10^{18}$ electrons.

\begin{equation} \label{eq:1}
e^- = -1.6 \cdot 10^{-19} [C]
\end{equation}

Using \ref{eq:1} for the charge of a single electron we let $Q_C$ be the total charge of the 2.5 trillion electrons and calculate that

\begin{align} \label{eq:2}
Q_C &= 2.5 \cdot 10^{18} \cdot e^- \\ &= -0.4 [C]
\end{align}

In order to calculate how much work it takes to move $Q_C$ from infinitely far away to point A,
we need to know how strong the electric field is at any point on the journey. 
Let $r$ be the distance from point A to any point in 3D space and $\epsilon_0$\footnote{$\epsilon_0 = 8.854 \cdot 10 ^ {-12} [m/F] $} the permitivity in empty space (vacuum),
then the electric field $E$ at any distance r from point A is 

\begin{align} \label{eq:3}
E &= \frac{Q_A}{4 \pi \epsilon_0 r^2}
\end{align}

And the force $F$ at any point is

\begin{align} \label{eq:4}
F &= E \cdot Q_C \\
  &= \frac{Q_A}{4 \pi \epsilon_0 r^2} \cdot Q_C
\end{align}

To get the work it takes to move $Q_C$ towards point A from distance $\infty$ to a distance 0 from point A,
we then take the integral of the force from $\infty$ to 0. Note how we introduce the concept of the mathematical limit
to be able to deal with infinities.

\begin{align} \label{eq:5}
W &= - lim_{x\to\infty} \int_x^0{\frac{Q_A Q_C}{4 \pi \epsilon_0 r^2}} \cdot dr \\
  &= - \frac{Q_A Q_C}{4 \pi \epsilon_0 } lim_{x\to\infty} \int_x^0{\frac{1}{r^2}} \cdot dr \\
  &= - \frac{Q_A Q_C}{4 \pi \epsilon_0 } lim_{x\to\infty} \left( \frac{1}{x} - \frac{1}{0} \right) \\
  &= - \frac{Q_A Q_C}{4 \pi \epsilon_0 } \left( 0 - \frac{1}{0} \right)
\end{align}

Now we find ourselves confronted with the infamous divide by 0 problem and it dawns on us, that bringing a charge
$Q_C$ on top on another charge $Q_A$, where the force between the two charges is known to be

\begin{align} \label{eq:6}
F &= \frac{Q_A  Q_C}{4 \pi \epsilon_0 r^2}
\end{align}

will lead to a infinitely large force pushing $Q_C$ away from $Q_A$. After checking that a integral of the form of

\begin{align} \label{eq:7}
\int_\infty^0{\frac{1}{r^2}} \cdot dr \\
\end{align}

indeed is divergant\footnote{ See example 8 here: http://tutorial.math.lamar.edu/Classes/CalcII/ImproperIntegrals.aspx \\ and result here: https://www.wolframalpha.com/input/?i=integral+of+1%2Fx%5E2+from+infinity+to+0
} we give in to the fact that electrons can not exist on top of each other and reinterpret "How much work is necessary to get
2.5 trillion electrons from infinitely far away to point A" as "How much work is necessary to get
2.5 trillion electrons from infinitely far away to very close to point A".

\begin{align} \label{eq:8}
W &= - lim_{x\to\infty} \int_x^{0.000000001[m]}{\frac{Q_A Q_C}{4 \pi \epsilon_0 r^2}} \cdot dr \\
  &= - \frac{Q_A Q_C}{4 \pi \epsilon_0 } lim_{x\to\infty} \int_x^{0.000000001[m]}{\frac{1}{r^2}} \cdot dr \\
  &= - \frac{Q_A Q_C}{4 \pi \epsilon_0 } lim_{x\to\infty} \left( \frac{1}{x} - \frac{1}{0.000000001[m]} \right) \\
  &= - \frac{Q_A Q_C}{4 \pi \epsilon_0 } \left( 0 - \frac{1}{0.000000001[m]} \right) \\
  &= - \frac{-1[C] \cdot -0.4[C]}{4 \pi \epsilon_0 } \left( - \frac{1}{0.000000001[m]} \right) \\
  &= - \frac{-1[C] \cdot -0.4[C]}{4 \pi 8.854 \cdot 10 ^ {-12} [F/m] } \left( - \frac{1}{0.000000001[m]} \right) \\
  &= \frac{0.4 \cdot 10 ^ {12} [C^2]}{4 \pi \cdot 8.854 [F]} \left( \frac{1}{0.000000001} \right) \\
  &= 3.595 \cdot 10^{18} \frac{[C^2]}{[F]} \\
  &= 3.595 \cdot 10^{18} [C \cdot V] \\
  &= 3.595 \cdot 10^{18} \left[A \cdot s \cdot \frac{N \cdot m}{A \cdot s}\right] \\
  &= 3.595 \cdot 10^{18} [N \cdot m] \\
  &= 3.595 \cdot 10^{18} [J]
\end{align}

We come to the conclusion that it takes $3.5950 \cdot 10^{18} [J]$ to get 2.5 trillion electrons from infinitely far away
to 1 [nm] near point A.


%------------------------------------------------

\end{document}
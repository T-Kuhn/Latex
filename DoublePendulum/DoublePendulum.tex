%%%%%%%%%%%%%%%%%%%%%%%%%%%%%%%%%%%%%%%%%
% Short Sectioned Assignment
% LaTeX Template
% Version 1.0 (5/5/12)
%
% This template has been downloaded from:
% http://www.LaTeXTemplates.com
%
% Original author:
% Frits Wenneker (http://www.howtotex.com)
%
% License:
% CC BY-NC-SA 3.0 (http://creativecommons.org/licenses/by-nc-sa/3.0/)
%
%%%%%%%%%%%%%%%%%%%%%%%%%%%%%%%%%%%%%%%%%

%----------------------------------------------------------------------------------------
%	PACKAGES AND OTHER DOCUMENT CONFIGURATIONS
%----------------------------------------------------------------------------------------

\documentclass[paper=a4, fontsize=11pt]{scrartcl} % A4 paper and 11pt font size

\usepackage[T1]{fontenc} % Use 8-bit encoding that has 256 glyphs
\usepackage{fourier} % Use the Adobe Utopia font for the document - comment this line to return to the LaTeX default
\usepackage[english]{babel} % English language/hyphenation
\usepackage{amsmath,amsfonts,amsthm} % Math packages
\usepackage{cancel}
\usepackage{graphicx}

\usepackage{lipsum} % Used for inserting dummy 'Lorem ipsum' text into the template

\usepackage{sectsty} % Allows customizing section commands
\allsectionsfont{\centering \normalfont\scshape} % Make all sections centered, the default font and small caps

\usepackage{svg}

\usepackage{siunitx}
\sisetup{detect-all}

\usepackage{fancyhdr} % Custom headers and footers
\pagestyle{fancyplain} % Makes all pages in the document conform to the custom headers and footers
\fancyhead{} % No page header - if you want one, create it in the same way as the footers below
\fancyfoot[L]{} % Empty left footer
\fancyfoot[C]{} % Empty center footer
\fancyfoot[R]{\thepage} % Page numbering for right footer
\renewcommand{\headrulewidth}{0pt} % Remove header underlines
\renewcommand{\footrulewidth}{0pt} % Remove footer underlines
\setlength{\headheight}{13.6pt} % Customize the height of the header

\numberwithin{equation}{section} % Number equations within sections (i.e. 1.1, 1.2, 2.1, 2.2 instead of 1, 2, 3, 4)
\numberwithin{figure}{section} % Number figures within sections (i.e. 1.1, 1.2, 2.1, 2.2 instead of 1, 2, 3, 4)
\numberwithin{table}{section} % Number tables within sections (i.e. 1.1, 1.2, 2.1, 2.2 instead of 1, 2, 3, 4)

\setlength\parindent{0pt} % Removes all indentation from paragraphs - comment this line for an assignment with lots of text

%----------------------------------------------------------------------------------------
%	TITLE SECTION
%----------------------------------------------------------------------------------------

\newcommand{\horrule}[1]{\rule{\linewidth}{#1}} % Create horizontal rule command with 1 argument of height

\title{
\normalfont \normalsize 
\horrule{0.5pt} \\[0.4cm] % Thin tp horizontal rule
\huge Simulating the Movement of a Double Pendulum with Euler's Method \\ % The assignment title
\horrule{2pt} \\[0.5cm] % Thick bottom horizontal rule
}

\author{Tobias Kuhn} % Name

\date{\normalsize\today} % Today's date or a custom date

\begin{document}

\maketitle % Print the title

\section{Important note}
This is a work in progress. It is not finished at this moment in time.

\section{What are we trying to do here?}
We want to spend some time thinking about double pendulums and numerical simulations.
Our goal is, of course, to create a working simulation of a double pendulum. 
What has to be done to get there? What kind of math is necessary?
Let's start our journey by breaking down the problem into smaller pieces.
\begin{itemize}
  \item Defining and labeling of the mathematical double pendulum model 
  \item Deriving the Differential Equation
  \item Setting up the simulation using the Differential Equation
\end{itemize}
That's better. The problem doesn't look all that daunting anymore.

\section{defining the double pendulum}

\section{Getting results with Euler's Method}
In this section we'll try to get ourselfes acquainted with Euler's Method.
We want to get some intuition as to how it is applied and what exactly it does.
Since we've never used this thing before, we're also curious to see it in action to confirm it's value as a mathematical tool.
\begin{align} 
  \dot{y} &= f(y, t) \label{eq:em1} \\
  y(t_0) &= y_0 \label{eq:em2}
\end{align}
Given an ordinary differential equation in the form of (\ref{eq:em1}) and with the initial conditions as shown in (\ref{eq:em2}), Euler's Method states that
\begin{align} \label{eq:em3}
  t_1 &= t_0 + \epsilon \\
  y_1 &= y_0 + \epsilon f(y_0, t_0)
\end{align}
Here, $\epsilon$ is a very small number. Though being small $\epsilon$ is not infinetely small at all.
Since we're planning on running this simulation on a computer, there are very real constraints to how small we are able to make this number.
Note how (\ref{eq:em3}) gives us the recipe for the first value pair $t_1$ and $y_1$. Any further value pairs $t_n$ and $y_n$ are calculated following the same pattern.
Note also how we started with a differential equation equalling a derivative $\dot{y}$ to a function of $y$ and $t$, but the result we get is $y$. Speaking physically,
if we started with a differential equation equalling the acceleration to a function of time and velocity, then Eulers Method would give us velocity, time pairs as a result.
\vspace{\baselineskip}

After spending some time thinking about Euler's Method, we are now confident that we got some intuition about how to use it and what results
to expect. But we still haven't used the damn thing. So let's give it a roll.
\begin{align} \label{eq:em4}
\frac{d}{dt} v(t) &= g - \frac{k}{m} \cdot v(t)^2
\end{align}
Here's a first order differential equation. 
And it's quite a special one, for it is a differential equation whose solution can be determined analytically. This is quite rare.
The differential equation modeling the double pendulum isn't quite as tame - There are no mathematical tools which could lead us to an analytically derived solution. 
But back to (\ref{eq:em4}). This equation models a falling object experiencing air resistance. The air resistance in this specific case is modeled to be proportional 
to the velocity squared. There are also quite a few constants in the mix. Let's go through them one by one.

$k$ is the air resistance coefficient and it's magnitude is
\begin{align} \label{eq:em5}
k &= 7.757 \cdot 10^{-6}[\si{\kilogram\per\meter}].
\end{align}
$g$ is the acceleration of gravity.
\begin{align} \label{eq:em6}
g &= 9.81 [\si{\meter\per\second\squared}]
\end{align}
and $m$ is the mass of the falling object.
\begin{align} \label{eq:em7}
m &= 70 [\si{\milli\gram}]
\end{align}
Furthermore we know that the solution to (\ref{eq:em4}) is
\begin{align}
v(t) &= \frac{1}{\beta} \cdot \frac{e^{2 g \beta t} - 1}{e^{2 g \beta t} + 1}.\label{eq:em8}
\end{align}
Where $\beta$ is defined as
\begin{align} 
\beta &= \sqrt{\frac{k}{gm}}. \\
      &= \sqrt{\frac{7.757 \cdot 10^{-6}[\si{\kilogram\per\meter}]}{9.81 [\si{\meter\per\second\squared}] \cdot 7 \cdot 10^{-5} [\si{\kilo\gram}]}} \\
      &= 0.106 [\si{\per\meter\second}]. \label{eq:em9}
\end{align}
Knowing all this allows us to put Euler's Method to the test. How will our numerically computed value for, for instance, $t_p = 10 \si{\second}$ measure up to the exact value 
computed by inserting $t_p = 10 \si{\second}$ into (\ref{eq:em8}). We might use $v_a(t)$ and $v_n(t)$ to differentiate between results gained from the analytical equation 
as opposed to results computed by the numerical approach. 

\begin{align}
v_a(t_p = 10\si{\second}) &= \frac{1}{\beta} \cdot \frac{e^{2 g \beta t_p} - 1}{e^{2 g \beta t_p} + 1}.\label{eq:em10} \\
 &= 9.434 \si{\meter\per\second}.\label{eq:em11}
\end{align}





\section{constructing the differential equation}
In order to get the simulation up and running, we need equations (14) and (19) from scienceworld's Double Pendulum page\footnote{http://scienceworld.wolfram.com/physics/DoublePendulum.html}.
Here they are.
\begin{align} \label{eq:1}
  (m_1 + m_2) l_1 \ddot{\theta}_1 + m_2 l_2 \ddot{\theta}_2 \cos(\theta_1 - \theta_2)
  + m_2 l_2 \dot{\theta}^2_2 \sin(\theta_1 - \theta_2) + g (m_1 + m_2) \sin \theta_1 &= 0
\end{align}
\begin{align} \label{eq:2}
  m_2 l_2 \ddot{\theta}_2 + m_2 l_1 \ddot{\theta}_1 cos(\theta_1 - \theta_2)
  - m_2 l_1 \dot{\theta}^2_1 sin(\theta_1 - \theta_2) + m_2 g \sin \theta_2 &= 0
\end{align}
Those two equations are both second order. In preparation for using Euler's Method on them, we need to split each of them into first order differential equations.
But how does one split them? We do this by introducting $\lambda_1$ and $\lambda_2$ which are defined as follows.
\begin{align} 
  \lambda_1 &= \dot{\theta}_1 \label{eq:3} \\
  \dot{\lambda}_1 &= \ddot{\theta}_1 \label{eq:4}
  \\
  \lambda_2 &= \dot{\theta}_2 \label{eq:5} \\
  \dot{\lambda}_2 &= \ddot{\theta}_2 \label{eq:6}
\end{align}
Putting these lambdas into (\ref{eq:1}) yields
\begin{align} \label{eq:7}
  (m_1 + m_2) l_1 \dot{\lambda}_1 + m_2 l_2 \dot{\lambda}_2 \cos(\theta_1 - \theta_2)
  + m_2 l_2 \lambda^2_2 \sin(\theta_1 - \theta_2) + g (m_1 + m_2) \sin \theta_1 &= 0.
\end{align}
And doing the same thing to (\ref{eq:2}) yields
\begin{align} \label{eq:8}
  m_2 l_2 \dot{\lambda}_2 + m_2 l_1 \dot{\lambda}_1 cos(\theta_1 - \theta_2)
  - m_2 l_1 \lambda^2_1 sin(\theta_1 - \theta_2) + m_2 g \sin \theta_2 &= 0.
\end{align}
We are in the process of preparing our differential equations for Euler's Method and the next crucial step is to rewrite (\ref{eq:7}) and (\ref{eq:8})
in such a way that both have a single $\dot{\lambda}$ on the left side of the equal sign. For (\ref{eq:7}) we get
\begin{align} \label{eq:9}
   \dot{\lambda}_1 &= \frac{- m_2 l_2 \dot{\lambda}_2 \cos(\theta_1 - \theta_2) - m_2 l_2 \lambda^2_2 \sin(\theta_1 - \theta_2) g (m_1 + m_2) \sin \theta_1}{(m_1 + m_2) l_1}.
\end{align}

\end{document}
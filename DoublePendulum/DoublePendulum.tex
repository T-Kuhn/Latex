%%%%%%%%%%%%%%%%%%%%%%%%%%%%%%%%%%%%%%%%%
% Short Sectioned Assignment
% LaTeX Template
% Version 1.0 (5/5/12)
%
% This template has been downloaded from:
% http://www.LaTeXTemplates.com
%
% Original author:
% Frits Wenneker (http://www.howtotex.com)
%
% License:
% CC BY-NC-SA 3.0 (http://creativecommons.org/licenses/by-nc-sa/3.0/)
%
%%%%%%%%%%%%%%%%%%%%%%%%%%%%%%%%%%%%%%%%%

%----------------------------------------------------------------------------------------
%	PACKAGES AND OTHER DOCUMENT CONFIGURATIONS
%----------------------------------------------------------------------------------------

\documentclass[paper=a4, fontsize=11pt]{scrartcl} % A4 paper and 11pt font size

\usepackage[T1]{fontenc} % Use 8-bit encoding that has 256 glyphs
\usepackage{fourier} % Use the Adobe Utopia font for the document - comment this line to return to the LaTeX default
\usepackage[english]{babel} % English language/hyphenation
\usepackage{amsmath,amsfonts,amsthm} % Math packages
\usepackage{cancel}
\usepackage{graphicx}

\usepackage{lipsum} % Used for inserting dummy 'Lorem ipsum' text into the template

\usepackage{sectsty} % Allows customizing section commands
\allsectionsfont{\centering \normalfont\scshape} % Make all sections centered, the default font and small caps

\usepackage{svg}

\usepackage{siunitx}
\sisetup{detect-all}

\usepackage{fancyhdr} % Custom headers and footers
\pagestyle{fancyplain} % Makes all pages in the document conform to the custom headers and footers
\fancyhead{} % No page header - if you want one, create it in the same way as the footers below
\fancyfoot[L]{} % Empty left footer
\fancyfoot[C]{} % Empty center footer
\fancyfoot[R]{\thepage} % Page numbering for right footer
\renewcommand{\headrulewidth}{0pt} % Remove header underlines
\renewcommand{\footrulewidth}{0pt} % Remove footer underlines
\setlength{\headheight}{13.6pt} % Customize the height of the header

\numberwithin{equation}{section} % Number equations within sections (i.e. 1.1, 1.2, 2.1, 2.2 instead of 1, 2, 3, 4)
\numberwithin{figure}{section} % Number figures within sections (i.e. 1.1, 1.2, 2.1, 2.2 instead of 1, 2, 3, 4)
\numberwithin{table}{section} % Number tables within sections (i.e. 1.1, 1.2, 2.1, 2.2 instead of 1, 2, 3, 4)

\setlength\parindent{0pt} % Removes all indentation from paragraphs - comment this line for an assignment with lots of text

%----------------------------------------------------------------------------------------
%	TITLE SECTION
%----------------------------------------------------------------------------------------

\newcommand{\horrule}[1]{\rule{\linewidth}{#1}} % Create horizontal rule command with 1 argument of height

\title{
\normalfont \normalsize 
\horrule{0.5pt} \\[0.4cm] % Thin tp horizontal rule
\huge Simulating the Movement of a Double Pendulum with Euler's Method \\ % The assignment title
\horrule{2pt} \\[0.5cm] % Thick bottom horizontal rule
}

\author{Tobias Kuhn} % Name

\date{\normalsize\today} % Today's date or a custom date

\begin{document}

\maketitle % Print the title

\section{Important note}
This is a work in progress. It is not finished at this moment in time.

\section{What are we trying to do here?}
We want to spend some time thinking about double pendulums and numerical simulations.
Our goal is, of course, to create a working simulation of a double pendulum. 
What has to be done to get there? What kind of math is necessary?
Let's start our journey by breaking down the problem into smaller pieces.
\begin{itemize}
  \item Defining and labeling of the mathematical double pendulum model 
  \item Deriving the Differential Equation
  \item Setting up the simulation using the Differential Equation
\end{itemize}
That's better. The problem doesn't look all that daunting anymore.

\section{defining the double pendulum}

\section{constructing the differential equation}
In order to get the simulation up and running, we need equations (14) and (19) from scienceworld's Double Pendulum page\footnote{http://scienceworld.wolfram.com/physics/DoublePendulum.html}.
Here they are.

\begin{align} \label{eq:1}
(m_1 + m_2) l_1 \ddot{\theta}_1 + m_2 l_2 \ddot{\theta}_2 \cos(\theta_1 - \theta_2)
 + m_2 l_2 \dot{\theta}^2_2 \sin(\theta_1 - \theta_2) + g (m_1 + m_2) \sin \theta_1 &= 0
\end{align}

\begin{align} \label{eq:1}
  m_2 l_2 \ddot{\theta}_2 + m_2 l_1 \ddot{\theta}_1 cos(\theta_1 - \theta_2)
  - m_2 l_1 \dot{\theta}^2_1 sin(\theta_1 - \theta_2) + m_2 g \sin \theta_2 &= 0
\end{align}











\end{document}